\documentclass[10pt]{extarticle}
% \documentclass[11pt]{article}
\usepackage{anyfontsize}

\usepackage[utf8]{inputenc}
\usepackage[T1]{fontenc}
\usepackage[slovene]{babel}
\usepackage{lmodern}
\usepackage{enumitem}
\usepackage{amsmath}
\usepackage{amssymb}
\usepackage[margin=0.2in,paperwidth=21cm, paperheight=29cm]{geometry}

\DeclareMathOperator{\Ima}{Im} 
\DeclareMathOperator{\Inn}{Inn} 
\DeclareMathOperator{\Sim}{Sim} 
\DeclareMathOperator{\Aut}{Aut} 


\title{Algebra 2}
\author{}
\date{}

\begin{document}
Množica $G$ skupaj z binarno operacijo $(x,y)\mapsto xy$ je grupa, če velja (za vse $x,y,z\in G$): 1)$(xy)z=x(yz)$; 2) obstaja enota/element $1\in G$, da je $1x=x1=x$; 3) obstaja inverz $x^{-1}\in G$, da je $xx^{-1}=x^{-1}x=1$. Če velja še $xy=yx$ je to Abelova grupa.


    \textbf{Primeri grup}

        
             Grupa, ki ima red 1, je trivialna grupa.
             Bijektivni preslikavi iz množice $X$ v množico $X$ pravimo permutacija množice $X$. Simetrična grupa $Sim(X)$ je grupa permutacij množice $X$.
             Splošna linearna grupa $GL_n(F)=\{A\in M_n(F)~|~detA\neq 0\}$ je grupa vseh obrnljivih matrik.
             Diedrska grupa $D_{2n}$ reda $2n$ je grupa simetrij previlnega $n$-kotnika. Velja: $zr=r^{-1}z=r^{n-1}z$ in $zr^k=r^{-k}z$.
             Množica $G_1\times \cdots \times G_s$ z vpeljano operacijo ('množenje po komponentah') je direktni produkt grup.

            Za neprazno podmnožico $H$ grupe $G$ veljajo ekvivalence: 1) $H$ je podgrupa $G$; 2) za vse $x,y\in H$ je $xy^{-1}\in H$; 3) $H$ je zaprta za množenje in za vsak $x\in H$ je $x^{-1}\in H$.
             Neprazna končna podmnožica $H$ grupe $G$ je podgrupa natanko tedaj, ko je zaprta za množenje.
             Podmnožica $H$ grupe $\mathbb{Z}$ je podgrupa za seštevanje natanko tedaj, ko je $H=n\mathbb{Z}$ za nek $n\in\mathbb{N}\cup\{0\}$.
             Produkt podgrup $H$ in $K$ grupe $G$ je $HK=\{hk~|~h\in H, k\in K\}$. Če je $HK=KH$ je $HK\leq G$.
             Posebna linearna grupa: $SL_n(F)=\{A\in M_n(F)~|~detA=1\}$.
             Ortogonalna grupa: $O_n=\{A\in M_n(\mathbb{R})~|~AA^t=I\}$.
             Unitarna grupa: $U_n=\{A\in M_n(\mathbb{C})~|~AA^*=I\}$.
             Center grupe $G$ je množica $Z(G)=\{c\in G~|~cx=xc~ za ~ \forall x\in G\}$.


             Naj bo $H$ podgrupa grupe $G$ in $a\in G$. Množica $aH=\{ah~|~h\in H\}$ je odsek grupe $G$ po podgrupi $H$.
             Za $a,b\in G$ velja: $aH=bH\Leftrightarrow a^{-1}b\in H$.
             Za $a,b\in G$ velja: odseka $aH$ in $bH$ sta bodisi enaka bodisi disjunktna.
            Moči množice vseh odsekov $\{aH~|~a\in G\}$ grupe $G$ po podgrupi $H$ pravimo indeks podgrupe $H$, oznaka $\left[G:H\right]$.
            (Lagrangeov izrek) Za podgrupo $H$ grupe $G$ je $|G|=\left[G:H\right]\cdot|H|$.
            Red vsake podgrupe končne grupe deli red grupe.


    \textbf{Grupe ostankov in ciklične grupe}
        
        
             Množica $\mathbb{Z}_n$ s seštevanjem $(a+n\mathbb{Z})+(b+n\mathbb{Z})=(a+b)+n\mathbb{Z}$ je Abelova grupa.
             Podgrupo $\langle a\rangle$ imenujem ciklična grupa, generirana z elementom $a$. Če je  $\langle a\rangle=G$ za kak $a\in G$, je $G$ ciklična grupa, $a$ je generator grupe $G$.
             $a\in G$. Če za $n\in \mathbb{N}$ velja $a^n=1$, potem rečemo, da ima $a$ končen red, najmanjšemu takemu številu $n$ pa red elementa $a$.
             Če ima element $a$ grupe $G$ končen red $n$, potem ima ciklična grupa $\langle a\rangle$ red $n$.
             Končna grupa reda $n$ je ciklična natanko tedaj, ko vsebuje element reda $n$.
             Red vsakega elementa končne grupe deli red grupe.
             Če je $G$ končna grupa, je $a^{|G|}=1$ za vsak $a\in G$.
             Vsaka grupa $G$ s praštevilskim redom je ciklična; za vsak od $1$ različen element $a\in G$ je $\langle a\rangle=G$.
             Vsaka ciklična grupa je Abelova.
        
             Najmanjšo podgrupo $G$, ki vsebuje podmnožico $X$, označimo z $\langle X \rangle$, in ji pravimo podgrupa, generirana z množico $X$.
             Če je $\langle X \rangle = G$, rečemo da je grupa $G$ generirana z množico $X$. Elementom $X$ pravimo generatorji grupe $G$, množici $X$ pa grupa generatorjev $G$.




    \textbf{Definicija kolobarja, obsega in polja}

        
             Množica $K$ skupaj z binarnima operacijama seštevanja $(x,y)\to x+y$ in množenja $(x,y)\to xy$ je kolobar, če velja: $K$ je Abelova grupa za seštevanje; $K$ je monoid za množenje; veljavnost distributivnostnih zakonov: $(x+y)z=xz+yz$ in $z(x+y)=zc+zy$.
             V poljubnem kolobarju za vse $x,y,z\in K$ velja: 1) $0x=x0=0$; 2) $(-x)y=x(-y)=-(xy)$; 3) $(x-y)z=xz-yz$, $z(x-y)=zx-zy$; 4) $(-x)(-y)=xy$; 5) $(-1)x=x(-1)=-x$. 
             Neničeln kolobar je obseg, če je vsak njegov neničeln element obrnljiv. Komutativen obseg imenujemo polje.
             Element $x$ kolobarja $K$ je delitelj niča, če $x\neq 0$ in če obstaja tak $y\neq 0$ iz $K$, da je $xy=0$ ali $yx=0$.
             Element kolobarja, ki je enak svojemu kvadratu je idempotent. Če je $e$ idempotent ($e^2=e$), potem je tudi $1-e$ idempotent.
             Element $a$ je nilpotenten element oz. nilpotent, če je $a^n=0$ za nek $n\in\mathbb{N}$.
             Komutativen kolobar brez deliteljev niča je cel kolobar.
             Obrnljiv element kolobarja ni delitelj niča. Obsegi so brez deliteljev niča.
             V kolobarju brez deliteljev niča velja pravilo krajšanja.


    \textbf{Definicija algebre}

        
             Naj bo $F$ polje. Množica $V$ skupaj z (notranjo) binarno operacijo seštevanja $(u,v)\to u+v$ in zunanjo binarno operacijo iz $F\times V$ v $V$, imenovano množenje s skalarji in označeno $(\lambda, v)\to \lambda v$, je vektorski prostor nad $F$, če velja (za $\forall \lambda,\mu \in F$ in $\forall u,v\in V$): 1) $V$ je Abelova grupa za seštevanje; 2) $\lambda(u+v)=\lambda u+\lambda v$; 3) $(\lambda+\mu)v=\lambda v+\mu v$; 4) $\lambda(\mu v)=(\lambda\mu)v$ in 5) $1v=v$.
             V vsakem vektorskem prostoru $V$ nad $F$ velja (za vsak $\lambda \in F$ in $v\in V$): 1) $\lambda 0=0$; 2) $0v=0$; 3) $\lambda v=0~\Rightarrow\lambda=0\vee v=0$ in 4) $(-\lambda)v=-(\lambda v)=\lambda(-v)$.
             Množica $A$ skupaj z binarnima operacijama seštevanja in množenja ter zunanjo operacijo množenja s skalarji se imenuje algebra nad $F$, če velja: 1) $A$ je za seštevanje in množenje s skalarji vektorski prostor nad $F$; 2) $A$ je kolobar za seštevanje in množenje in 3) za vse $\lambda\in F$ in $x,y\in A$ je $\lambda(xy)=(\lambda x)y=x(\lambda y)$.
             Če je element algebre $a$ obrnljiv, potem ni delitelj niča.
             Kvaternioni: $\mathbb{H}=\{\lambda_01+\lambda_1i+\lambda_2j+\lambda_3k | \lambda_i\in\mathbb{R}\}$ ($4$-razsežna algebra, obseg, ni polje /ni komutativno/).
             Kvaternionska grupa: $Q=\{\pm 1,\pm i, \pm j, \pm k\}$


    \textbf{Podkolobarji, podalgebre in podpolja}

        
             Podmnožica $L$ kolobarja $K$ je podkolobar kolobarja $K$, če vsebuje enoto $1$ kolobarja $K$ in je za isti operaciji tudi sama kolobar.
            Podobno podalgebre, podprostori, podpolja.   
            Polje $E$ je razširitev polja $F$, če je $F$ podpolje $E$.
             Podmnožica $L$ kolobarja $K$ je podkolobar natanko tedaj, ko velja: 1) $1\in L$; 2) $L$ je podgrupa za seštevanje; 3) $L$ je zaprta za množenje.
             Podmnožica $B$ algebre $A$ je podalgebra natanko tedaj, ko velja: 1) $1\in B$, 2) $B$ zaprta za seštevanje; 3) $B$ zaprta za množenje; 4) za vsak skalar $\lambda$ in $x\in B$ je $\lambda x\in B$.
             Podmnožica $F$ polja $E$ je podpolje natanko tedaj, ko velja: 1) $1\in F$, 2) $F$ je podgrupa za seštevanje; 3) $F$ je zaprta za množenje; 4) za vsak $x\neq 0$ iz $F$ tudi $x^{-1}\in F$.
             Center kolobarja: $Z(K)=\{c\in K~|~xc=cx~za~vsak~x\in K\}$.


    \textbf{Kolobarji ostankov in karakteristika kolobarja}

        
             Naj bo $K$ kolobar. Ko obstajajo taka naravna števila $n$, da je $n\cdot 1=0$, potem najmanjšemu izmed njih pravimo karakteristika kolobarja $K$. V tem primeru ima $K$ končno karakteristiko. Če takih števil ni, rečemo, da ima $K$ karakteristiko $0$.
             Za kolobar $K$ s karakteristiko $n>0$ velja: 1) $nx=0$ za vse $x\in K$; 2) za vsak $m\in\mathbb{Z}$ je $m\cdot 1=0$ natanko tedaj, ko $n\mid m$; 3) če $K$ nima deliteljev niča in $K\neq\{0\}$ je $n$ praštevilo.
             Če v aditivno grupo $\mathbb{Z}_n$ vpeljemo množenje s predpisom $(a+n\mathbb{Z})\cdot(b+n\mathbb{Z})=ab+n\mathbb{Z}$, postane $\mathbb{Z}_n$ komutativen kolobar.
             Končen cel kolobar je polje.
             Naravno število $p$ je praštevilo natanko tedaj, ko je kolobar $\mathbb{Z}_p$ polje.
             (Wedderburnov izrek) Vsi končni obsegi so komutativni.
             (Fermatov mali izrek) Za vsako praštevilo $p$ in vsako naravno število $a$ je $a^p\equiv a\mod{p}$.
             (Eulerjev izrek) $k,n\in \mathbb{N}$: $\phi(n)=k$, kjer je $k<n; GCD(n,k)=1$, tu $k$ predstavlja število vseh tujih števil $n$ manjših od $n$. (Za elemente grupe obrnljivih elementov velja: $red(a)\mid\phi(n)$ in $a^{\phi(n)}=1\mod{n}$)


    \textbf{Generatorji kolobarjev, algeber in polj}

        
             Naj bo $K$ kolobar in $X$ njegova podmnožica. Podkolobar, generiran z $X$ je najmanjši kolobar, ki vsebuje $X$. Enak je preseku vseh podkolobarjev, ki  vsebujejo $X$. Če $X$ sestoji iz elementov $x_i$, rečemo da je $\overline{X}$ podkolobar, generiran z elementi $x_i$. Kadar je $\overline{X}=K$, rečemo, da je $K$ generiran z množico $X$ oziroma, da so elementi $X$ njegovi generatorji. Kolobar je končno generiran, če je generiran s končno množico.
             Podkolobar, generiran z $X$ je množica vseh elementov oblike $k_1x_{11}\cdots x_{1m_1}+k_2x_{21}\cdots x_{2m_2}+\cdots+k_nx_{n1}\cdots x_{nm_n}$, kjer $x_{ij}\in X\cup\{1\}$ in $k_i\in\mathbb{Z}$.
             Podalgebra, generirana z $X$ je množica vseh elementov oblike $\lambda_1x_{11}\cdots x_{1m_1}+\lambda_2x_{21}\cdots x_{2m_2}+\cdots+\lambda_nx_{n1}\cdots x_{nm_n}$, kjer $x_{ij}\in X\cup\{1\}$ in $\lambda_i\in F$.
             Podpolje, generirano z $X$ je množica vseh elementov oblike $uv^{-1}$, kjer je $u,v\in\overline{X}$ in $v\neq 0$.


    \textbf{Pojem homomorfizma}

        
             (D) Preslikava $\phi: A\to A'$ je homomorfizem grup, če sta $A$ in $A'$ grupi in za vse $x,y\in A$ velja $\phi(xy)=\phi(x)\phi(y)$.
             (D) Preslikava $\phi: A\to A'$ je homomorfizem algeber, če sta $A$ in $A'$ algebri nad istim poljem $F$ in za vse $x,y\in A$ in $\lambda \in F$ velja: $\phi(x+y)=\phi(x)+\phi(y)$; $\phi(xy)=\phi(x)\phi(y)$; $\phi(\lambda x)=\lambda\phi(x)$; $\phi(1)=1$.
             (T) Če je $\phi: A\to A'$ homomorfizem grup, je $\phi(1)=1$ in $\phi(x^{-1})=\phi(x)^{-1}$ za vse $x\in A$.
             (T) Če je $\phi: A\to A'$ homomorfizem aditivnih grup, je $\phi(0)=0$ in $\phi(-x)=-\phi(x)$ za vse $x\in A$.
             (T) Če je $\phi: A\to A'$ homomorfizem kolobarjev in je element $x\in A$ obrnljiv, je obrnljiv tudi $\phi(x)$ in velja $\phi(x^{-1})=\phi(x)^{-1}$.
             (T) Slika homomorfizma grup/prostorov/kolobarjev/algeber je podgrupa/podprostor/podkolobar/podalgebra.
             (T) Homomorfizem $\phi: A\to A'$ je injektiven natanko tedaj, ko je njegovo jedro trivialno (vsebuje le enoto $1$/$0$).
             (T) Kompozitum homomorfizmov je homomorfizem.
             (T) Inverzna preslikava izomorfizma je izomorfizem.
             (P) Množica vseh avtomofizmov $A$ je za operacijo komponiranja grupa.
             (D) Če obstaja izomorfizem iz $A$ v $A'$, sta $A$ in $A'$ izomorfna/-i $A\cong A'$.
             (T) Končno-razsežna vektorska prostora $V$ in $V'$ nad poljem $F$ sta izomorfna natanko tedaj, ko imata enako dimenzijo.
             (P) Netrivialen končno-razsežen vektorski prostor nad poljem $F$ je izomorfen prostoru $F^n$ za neki $n\in\mathbb{N}$.
             Če je v homomorfizmu $red(a)=k$, potem velja $\phi(a)^k=\phi(a^k)=\phi(1)=1$, od tod sledi $red(\phi(a))\mid red(a)$.
             (p) Notranji avtomofizem poljubne grupe $G$, za vsak $a\in G$, definiramo s preslikavo $\phi_a:G\to G$ s predpisom $\phi_a(x)=axa^{-1}$.
     

    \textbf{Podgrupe edinke in kvocientne grupe}

        
             (D) Za vsako podgrupo $N$ grupe $G$ je množica $aNa^{-1}=\left\{ana^{-1}~|~n\in\mathbb{N}\right\}$ podgrupa $G$, imenovana konjugirana podgrupa podgrupe $N$.
             (D) Če podgrupa $N$ grupe $G$ zagošča pogoju $aNa^{-1}\subseteq N$ za vsak $a\in G$, se imenunje (podgrupa) edinka: $N\triangleleft G$.
             Velja: $N\triangleleft G \Leftrightarrow N\leq G$ in $aNa^{-1}\subseteq N$ za vsak $a\in G$.
             (p) Trivialna podgrupa $\left\{1\right\}$ in cela grupa $G$ sta edinki vsake grupe $G$. ~~ 
             Vsaka podgrupa v Abelovi grupi je edinka. ~~ 
             Center $Z(G)$ vsake grupe $G$ je edinka.            
             Netrivialna edinka -- vsaka edinka različna od $\left\{1\right\}$; prava edinka -- vsaka edinka različna od $G$.
             (D) Enostavna grupa je netrivialna grupa, ki nima pravih netrivialnih edink.
             (T) Za podgrupo $N$ grupe $G$ je ekvivalentno: 1) $N$ je edinka.; 2) $aN\subseteq Na$ za $\forall a\in G$.; 3) $aN=Na$ za $\forall a\in G$.; 4) $aNa^{-1}=N$ za $\forall a\in G$.
             Produkt podgrupe z edinko je podgrupa. Produkt edink je edinka.; $H\leq G, N\triangleleft G \Rightarrow HN=NH\leq G$.; $M,N\triangleleft G \Rightarrow MN=NM\triangleleft G$.; $M,N\triangleleft G \Rightarrow M\cap N\triangleleft G$.
             (I) Naj bo $N\triangleleft G$. Če v množico vseh odsekov $G/N$ vpeljemo množenje s predpisom $aN\cdot bN=(ab)N$, postane $G/N$ grupa. Preslikava $\pi:G\to G/N$, definirana s $\pi(a)=aN$, je epimorfizem in $\ker \pi=N$.
             (D) Grupi $G/N$ pravimo kvocientna grupa (tudi faktorska grupa), preslikavo $\pi$ imenujemo kanonični epimorfizem.
             (O) Če je $G$ aditivna grupa, operacijo v kvocientni grupi $G/N$ vpeljemo kot seštevanje $(a+N)+(b+N)=(a+b)+N$.; Množenje odsekov $aN\cdot bN=(ab)N$ je dobro definirano.
             (O) Če je $G$ končna grupa in $N$ njena poljubna edinka, je po Lagrangevem izreku $|G/N|=\frac{|G|}{|N|}$.
             (T) Podmnožica $N$ grupe $G$ je podgrupa edinka natanko tedaj, ko je $N$ jedro homomorfizma iz grupe $G$ v neko grupo $G'$.
             (I) Naj bo $U$ podprostor vektorskega prostora $V$. Če v množico vseh odsekov $V/U$ vpeljemo seštevanje in množenje s skalarji s predpisi $(v+U)+(w+U)=(v+w)+U$ in $\lambda(v+U)=\lambda v+U$, postane $V/U$ vektorski prostor. Preslikava $\pi: V\to V/U$, s predpisom $\pi(v)=v+U$ je epimorfizem in $\ker\pi=U$. (kvocientni vektorki prostor; kanonični epimorfizem)

    \textbf{Ideali in kvocientni kolobarji}

        
             (D) Naj bo $I$ podgrupa kolobarja $K$ za seštevanje. Če za vse $a \in K$ in $u\in I$ velja $au\in I$ in $ua\in I$, $I$ imenujemo ideal kolobarja $K$: $I\triangleleft K$. 
             $I$ ideal kolobarja $K$: 1) $u-v\in I$ za $\forall u,v\in I$; 2) $KI\subseteq I$; 3) $IK\subseteq I$. ((1 in 2) - levi ideal; (1 in 3) - desni ideal)
             (p) $\left\{0\right\}$ in $K$ ideala kolobarja $K$. ~~ Glavni ideali: $aK=\left\{ax~|~x\in K\right\}$. ~~ $n\mathbb{Z}; n\in\mathbb{N}\cup\left\{0\right\}$ glavni ideali za $\mathbb{Z}$.
             (L) Če enostranski ali dvostranski ideal $I$ kolobarja $K$ vsebuje kak obrnljiv element, je enak celemu kolobarju $K$.
             (D) Vsota idealov $I$ in $J$ kolobarja $K$: $I+J=\left\{u+v~|~u\in I, v\in J\right\}$. Vsota je ideal. Produkt idealov $I$ in $J$ kolobarja $K$: $IJ=\left\{u_1v_1+\cdots+u_nv_n~|~u_i\in I, v_i\in J\right\}$. Produkt je ideal. Presek idealov $I$ in $J$ kolobarja $K$ je ideal kolobarja $K$.
             (I) Naj bo $I\triangleleft K$. Če v množico vsek odsekov $K/I$ vpeljemo seštevanje in množenje s predpisi $(a+I)+(b+I)=(a+b)+I$ in $(a+I)(b+I)=ab+I$, postane $K/I$ kolobar. Preslikava $\pi: K\to K/I$, s predpisom $\pi(a)=a+I$, je epimorfizem in $\ker\pi =I$. (kvocientni kolobar; kanonični epimorfizem)
             (O) Podgrupa za seštevanje $I$ kolobarja $K$ je ideal, ko je množenje odsekov $(a+I)(b+I)=ab+I$ dobro definirano.
             (T) Podmnožica $I$ kolobarja $K$ je ideal natanko tedaj, ko je $I$ jedro homomorfizma iz kolobarja $K$ v nek kolobar $K'$.
             (D) Ideal algebre definiran kot ideal kolobarja. Je podprostor, ker je za $\forall \lambda$ in $\forall u$ tudi $\lambda u=\lambda(1u)=(\lambda 1)u$ element ideala.
             (I) Naj bo $I$ ideal algebre $A$. Če v množico vseh odsekov $A/I$ vpeljemo seštevanje, množenje in množenje s sklarji s predpisi $(a+I)+(b+I)=(a+b)+I$, $(a+I)(b+I)=ab+I$ in $\lambda(a+I)=\lambda a+I$, postane $A/I$ algebra. Preslikava $\pi:A\to A/I$, s predpisom $\pi(a)=a+I$, je epimorfizem in $\ker\pi=I$. (kvocientna algebra, kanonični epimorfizem.)

    \textbf{Izrek o izomorfizmu in primeri kvocientnih struktur}

        
             (I - prvi izrek o izomorfizmu) Naj bo $\phi:A\to A'$ homomorfizem. Potem je $A/\ker\phi\cong\Ima\phi$.
             (I) Vsaka ciklična grupa je izomorfna bodisi grupi $\mathbb{Z}$ bodisi grupi $\mathbb{Z}_n$ za nek $n\in\mathbb{N}$.
             (P) Netrivialna grupa $G$ nima pravih netrivialnih podgrup natanko tedaj, ko je $G$ ciklična grupa s praštevilskim redom (in je torej $G\cong\mathbb{Z}_p$ za nek $p\in\mathbb{P}$).
             (p) Velja: $G/\left\{1\right\}\cong G$ in $G/G\cong\left\{1\right\}$. ~~ $G/Z(G)\cong\Inn(G)$. ($\Inn(G)$ grupa notranjih avtomorfizmov).
             (O) Center $Z(G)$ vsake grupe $G$ je edinka, center $Z(K)$ nekomutativnega kolobarja $K$ ni ideal (vsebuje enoto $1$).
             (D) Idealu $M$ kolobarja $K$ pravimo maksimalni ideal, če $M\neq K$ in če ne obstaja ideal $J$ z lastnostjo $M\subsetneq J\subsetneq K$.
             (I) Ideal $M$ komutativnega kolobarja $K$ je maksimalni ideal natanko tedaj, ko je kvocientni kolobar $K/M$ polje.
             (p) Velja: $K/\left\{0\right\}\cong K$ in $K/K\cong\left\{0\right\}$.
             (p) Za naravno število $p$ je ekvivalentno: 1) $p\in\mathbb{P}$; 2) Kolobar $\mathbb{Z}_p$ je polje.; 3) $p\mathbb{Z}$ je maksimalni ideal kolobarja $\mathbb{Z}$.
             (I - drugi izrek o izomorfizmu) Naj bo $G$ grupa, $H\leq G$ in $N\triangleleft G$. Velja: $H\cap N \triangleleft H$, $N\triangleleft HN$ in $H/(H\cap N)\cong HN/N$.
             (I - tretji izrek o izomorfizmu) Naj bo $G$ grupa, $M,N\triangleleft G$ in $N\subseteq M$. Velja: $G/M\cong (G/N)/(M/N)$.

    \textbf{Korespondenčni izrek}

        
             (L) Naj bo $\phi:G\to G'$ homomorfizem grup. 1) Če je $H'\leq G'$ je $\phi^{-1}(H')\leq G$.; 2) Če je $N'\triangleleft G'$ je $\phi^{-1}(N')\triangleleft G$.; 3) Če je $H\leq G$ je $\phi(H)\leq G'$.; 4) Če je $N\triangleleft G$ in je $\phi$ epimorfizem, je $\phi(N)\triangleleft G'$.
             (I - korespondenčni izrek) Naj bo $N\triangleleft G$. 1) Vsaka podgrupa grupe $G/N$ je oblike $H/N$ za neko podgrupo $H$ grupe $G$, ki vsebuje $N$.; 2) Vsaka podgrupa edinka grupe $G/N$ je oblike $M/N$ za neko podgrupo edinko $M$ grupe $G$, ki vsebuje $N$.
             (P) Vsaka podgrupa grupe $\mathbb{Z}_n$ je oblike $d\mathbb{Z}_n$, kjer $d\mid n$.

\end{document}